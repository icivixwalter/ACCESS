
%% TUTORIAL LATEX = http://wwwcdf.pd.infn.it/AppuntiLinux/latex_introduzione.htm

%% Impaginazione come articolo, carta A4, caratter 10 punti twoside= fronte/retro
\documentclass[10pt,a4paper]{article}

%%  SORGENTE IN ITALIANO, ESATTA SEQUENZA DI DI PACCHETTI DA CARICARE.
%%	Imputec =	SIMBOLI LATINI +  fontec = LETTERE ACCENTATE E SIMBOLI SPECIALI + lingua principale 
%%  del documento = [italian]{babel}	
%%---------------------------------------------------------------------------------------------
%%L'esempio mostra l'inclusione del pacchetto inputenc allo scopo di ammettere 
%%la codifica dei caratteri ISO 8859-1 (Latin 1) nel sorgente LaTeX, assieme al pacchetto fontenc 
%%per ottenere una composizione con un tipo di carattere che contenga le lettere accentate
%% e i simboli speciali più importanti utilizzati in Europa. La lingua ufficiale del 
%% documento � l'italiano con babel.

\usepackage[latin1]{inputenc}						%Imputec = interpreta correttamente i caratteri inseriti
\usepackage[T1]{fontenc}							%t1= font usato per l'italiano
\usepackage[italian]{babel}							%babel= font usato per l'italiano	
%%---------------------------------------------------------------------------------------------



\usepackage{amsmath}
\usepackage{amsfonts}
\usepackage{amssymb}
\usepackage{graphicx}
\usepackage{hyperref}			%---------------------> ipertesto
\usepackage{xcolor}				%---------------------> colore iperlink
\usepackage{array,booktabs}
\usepackage{csvsimple, longtable, booktabs}			%---------------------> importa csv


%%//			PREAMBOLO IMPOSTAZIONE DEI MARGINI
%%//=========================================================================================//
%%				impostazione di margini base
%%------------------------------------------------------------
%\setlength{\textwidth}{16 cm}       %16cm  LUNGHEZZA RIGHE.
%\setlength{\oddsidemargin}{0 cm}    %0cm   MARGINE SINISTRO.
%\setlength{\topmargin}{0 cm}        %0cm   MARGINE SUPERIORE.
%\setlength{\textheight}{22 cm}      %22cm  LUNGHEZZA COLONNE.
%%------------------------------------------------------------

% modifiche rispetto ai margini base
\setlength{\textwidth}{16 cm}      	 	%16cm  LUNGHEZZA RIGHE.
\setlength{\oddsidemargin}{-2 cm}    	%0cm   MARGINE SINISTRO.
\setlength{\topmargin}{-3 cm}          	%0cm   MARGINE SUPERIORE.
\setlength{\textheight}{20 cm}      	%22cm  LUNGHEZZA COLONNE.

%--------------------> per eliminare il numero di pagina inferiore
%\pagenumbering{empty}   %DA SCOMMENTARE SE NON SI VUOLE IL NUMERO DI PAGINA.
%%//=========================================================================================//


%//						MODULO_02
%//============================================================================================//
%//						MODULO_01
%//------------------------------------------------------------------------------------------//
%//---------------------> definizione??



\hypersetup{%
	colorlinks=false,% hyperlinks will be black
	linkbordercolor=red,% hyperlink borders will be red
	pdfborderstyle={/S/U/W 1}% border style will be underline of width 1pt
}

%//--------------------->PREAMBOLO DEL DOCUMENTO
\begin{document}
	\textbf{\underline{	APERTURA FORM DETERMINE}}
	
	

\part{ANALISI TREE}
	Gli archivi della gestione delle  sono valorizzati in :\\
	La path di archivio:\\
	\detokenize{C:\GESTIONI\GESTIONE_LLPP\25_GESTIONE_LLPP\LLPP_ARCHVI_MDB\LLPP_GESTIONE\LPP_ANALISI\DETERMINE}\\
	\href{run:e:/CASA/LINGUAGGI/ACCESS/ACCESS_PROGETTI_MDB/ACCESS_GE_OGGETTI_NEW/GE_OGGETTI_NEW_TEX+TXT_ANALISI/}{ ... APRI ANALISI}
	\label{cartella:Determine-Originale}
	\\
	Il tex per l'analisi delle Determine \\
	\detokenize{PRES3000_ANALISI_GESTIONE_Determine.tex}\\
	%%----------------> \href{run:C:....} attenzione la path deve essere completa con barra rovesciata. Meglio se metti una bat
	\href{run:C:/GESTIONI/GESTIONE_LLPP/25_GESTIONE_LLPP/LLPP_ARCHVI_MDB/LLPP_GESTIONE/LPP_ANALISI/DETERMINE/DETERMINE_ANALISI.tex}
	{ ... IL TEX DI ANALISI Determine}\\
	\label{cartella:Determine-Text}
	\\
	Tutti i file e le cartelle per l'analisi delle Determine \\
	\href{run:C:/GESTIONI/GESTIONE_LLPP/25_GESTIONE_LLPP/LLPP_ARCHVI_MDB/LLPP_GESTIONE/LPP_ANALISI/DETERMINE/DETERMINE_APRI_OGGETTI_DATABASE.bat}
	{... TUTTI I FILE DELLE DETERMINE APRI .BAT}
	\label{file:Determine-Bat}
	%APRI TUTTI I FILE E CARTELLE DELLE Determine CON IL BAT
		
	\part{APERTURA FORM**}
	....
	Apertura della form vengono resettate tutte le tab
	con il seguente codice:
	%//-------------------------------> inserisco il file di codice
	%\		
	%	\href{run: C:/TMP/appoggio.pdf}{APRI appoggio}
	
	% e' possibile usare path relativi: ./ e' la cartella corrente, ../ e' la directory precedente	
	
	%\	href{run:C:\TMP\appoggio.pdf}{apri}
	\href{run:C:/TMP/appoggio.pdf}{This is my link} 
	
	%// RIFERIMENTO A FAILE .TXT
	%//----------------------------------------------------------------------------------------------------------//
	%NOTE : L'indirizzo completo è il seguente, %c:\GESTIONI\GESTIONE_LLPP\25_GESTIONE_LLPP\LLPP_ARCHVI_MDB\LLPP_GESTIONE\LPP_ANALISI\DETERMINE\DETERMINE_Mdl02_ROUTINE_CODICE_APERTURA_FORM_(LPDD_DETERMINE).txt ma
	%ma per evitare errori di caricamento è stato dato solo quello relativo ./DETERMINE_Mdl02_ROUTINE_CODICE_APERTURA_FORM_(LPDD_DETERMINE).txt
	\href{run:./DETERMINE_Mdl02_ROUTINE_CODICE_APERTURA_FORM_(LPDD_DETERMINE).txt}{Modulo Apri form} 
	
	
	%//----------------------------------------------------------------------------------------------------------//
	
	\subsection{EVENTO FORM LOAD}

	
	


		
	
\end{document}